\subsection{Motivation: the factor crowding problem}

While factor investing generates systematic excess returns, empirical evidence reveals a troubling reality: alpha from factors decays over time. Hua and Sun (2020) document that as more capital flows into factor strategies, returns compress. DeMiguel, Garlappi, and Uppal (2020) quantify the magnitude: a one-standard-deviation increase in crowding reduces annualized returns by 8 percentage points---economically enormous. Yet surprisingly, the financial literature documents this phenomenon empirically without providing \textbf{mechanistic explanation}. We observe that crowding reduces returns; we lack the theory explaining why and when.

This gap creates three interconnected problems:

\textbf{(1) Decay Mechanism}: Factors decay at different rates, but which factors and why? Momentum decays faster than value; judgment factors faster than mechanical factors. No prior theory explains these patterns from first principles. Without mechanistic understanding, managers cannot forecast decay and are left reacting after crowding occurs.

\textbf{(2) Global Transfer}: Do US crowding dynamics apply globally? Practitioners need guidance on factor transferability across markets, but generic machine learning methods ignore market regime structure. Standard domain adaptation forces incompatible regimes (US bull vs. UK bear) to align uniformly, degrading transfer.

\textbf{(3) Risk Management}: Factor crashes are correlated with high crowding, yet existing risk models treat these as uncorrelated tail events. Current tools do not leverage crowding signals for dynamic hedging. This leaves portfolios vulnerable to crowding-driven crashes predictable ex-ante.

\subsection{Our contributions}

We propose a unified framework addressing all three problems:

\textbf{Contribution 1: Game-Theoretic Model of Crowding Decay (Section 4)} \\
We derive mechanistic explanation from first principles. Modeling investors as strategic agents, we show that rational exit timing generates hyperbolic decay: $\alpha_i(t) = K_i / (1 + \lambda_i t)$. The model predicts judgment factors decay faster than mechanical factors. Empirical validation on Fama-French data (1963-2024) confirms: $\lambda_{\text{judgment}} = 0.173 \pm 0.025$ vs. $\lambda_{\text{mechanical}} = 0.072 \pm 0.010$ ($p < 0.001$). This enables practitioners to forecast decay and time rotations.

\textbf{Contribution 2: MMD-Based Domain Adaptation for Global Transfer (Section 6)} \\
We apply Maximum Mean Discrepancy (MMD) domain adaptation to transfer US factor crowding insights globally. By aligning feature distributions between source (US) and target markets, MMD accounts for economic differences while preserving predictive power. On 7 developed markets, MMD achieves 60\% transfer efficiency vs. 43\% naive transfer. This enables confident global transfer of US crowding insights without independent local research.

\textbf{Contribution 3: Crowding-Weighted Conformal Prediction (Section 7)} \\
We integrate crowding signals into distribution-free uncertainty quantification. CW-ACI produces narrower sets during low crowding (high confidence) and wider sets during high crowding (high uncertainty), while preserving statistical coverage guarantees. On factor portfolios, CW-ACI-based hedging improves Sharpe ratio by 54\% (0.67 → 1.03) and reduces losses by 60-70\% during crashes.

These contributions are unified: game theory explains mechanism → domain adaptation enables global transfer → conformal prediction manages risk. This integration is novel; prior work addresses problems in isolation.

\subsection{Paper organization}

Section 2 reviews related literature. Section 3 provides background (notation moved here). Sections 4--7 develop the three contributions with theory and empirical validation. Section 8 discusses robustness. Section 9 concludes. Appendices contain proofs and reproducibility materials.
