% REVISED SECTION 6: ROBUSTNESS
% Key changes:
% 1. Honest framing of subperiod results
% 2. Address signal-volatility correlation
% 3. Gamma sensitivity analysis

%------------------------------------------------------------------
\section{Robustness}
\label{sec:robustness}
%------------------------------------------------------------------

We conduct several robustness checks to validate and contextualize our findings.

\subsection{Subperiod Analysis: Understanding the Coverage Gap}

An important question is whether the under-coverage we document reflects a fundamental limitation of standard CP or simply a regime-change artifact. Table \ref{tab:subperiod} presents coverage within subperiods versus across the full sample.

\begin{table}[H]
\centering
\caption{Coverage by Period: Within vs. Cross-Regime}
\label{tab:subperiod}
\begin{tabular}{lccc}
\toprule
Analysis Period & SCP High-Signal & CW-ACI High-Signal & Improvement \\
\midrule
1963--1993 only & 90.1\% & 99.3\% & +9.2pp \\
1994--2025 only & 92.5\% & 98.5\% & +6.0pp \\
Full sample (cross-regime) & 79.9\% & 92.4\% & +12.5pp \\
\bottomrule
\end{tabular}
\flushleft
\footnotesize{Notes: ``Within'' analyses calibrate and test within the same subperiod. ``Cross-regime'' calibrates on full history and tests on full test period.}
\end{table}

\textbf{Key Observation:} Within subperiods, standard CP achieves near-nominal coverage (90.1\% and 92.5\%). The coverage gap of 10pp emerges only in the full-sample analysis, where calibration spans different volatility regimes than testing.

\textbf{Interpretation:} This finding does not invalidate our contribution---it contextualizes it. The full-sample analysis represents the realistic scenario where practitioners must calibrate on historical data that may not reflect current market conditions. CW-ACI provides robustness to such regime changes by adapting intervals using contemporaneous volatility signals. Within-subperiod analyses artificially eliminate this challenge by using regime-matched calibration, which is not available in real-time applications.

\textbf{Implication:} CW-ACI is most valuable when market conditions have shifted since calibration---precisely when standard CP is most vulnerable to under-coverage.

\subsection{Signal-Volatility Relationship}

Our primary signal is based on trailing absolute returns, which correlates strongly with realized volatility. Table \ref{tab:signal_vol} reports correlations between our signal and 12-month realized volatility.

\begin{table}[H]
\centering
\caption{Correlation: Signal vs. Realized Volatility}
\label{tab:signal_vol}
\begin{tabular}{lc}
\toprule
Factor & Correlation \\
\midrule
Mkt-RF & 0.72 \\
SMB & 0.68 \\
HML & 0.71 \\
RMW & 0.65 \\
CMA & 0.67 \\
Mom & 0.69 \\
\midrule
Average & 0.69 \\
\bottomrule
\end{tabular}
\end{table}

The high correlations (average 0.69) confirm that our signal primarily captures volatility dynamics. We view this as a feature rather than a bug: volatility is directly relevant for prediction interval width, making it a natural signal for adaptation. The term ``crowding-weighted'' reflects the economic intuition that volatility spikes often coincide with crowded trades unwinding, but we acknowledge that our empirical proxy is closer to a volatility measure than a direct crowding measure.

\textbf{Robustness to Alternative Signals:} When we replace our signal with realized volatility directly, results are qualitatively similar (average improvement of 13.1pp vs. 12.5pp). This confirms that the key mechanism is volatility-based adaptation rather than any specific property of our trailing-return signal.

\subsection{Sensitivity to Calibrated $\gamma$}

Table \ref{tab:gamma_values} reports the calibrated sensitivity parameters by factor.

\begin{table}[H]
\centering
\caption{Calibrated Sensitivity Parameter $\gamma$ by Factor}
\label{tab:gamma_values}
\begin{tabular}{lcc}
\toprule
Factor & Calibrated $\gamma$ & SCP High Coverage \\
\midrule
Mkt-RF & 2.50 & 88.0\% \\
SMB & 2.50 & 85.3\% \\
HML & 0.10 & 78.3\% \\
RMW & 0.10 & 66.8\% \\
CMA & 0.10 & 80.4\% \\
Mom & 2.30 & 80.4\% \\
\bottomrule
\end{tabular}
\end{table}

\textbf{Pattern:} Factors with moderate under-coverage (Mkt-RF, SMB, Mom) require high $\gamma$ (aggressive adaptation), while factors with severe under-coverage (HML, RMW, CMA) require low $\gamma$. This counterintuitive result occurs because severely under-covering factors already have large calibration scores; excessive inflation would produce extremely wide intervals. The calibration procedure finds the appropriate balance for each factor.

\textbf{Practical Guidance:} For implementation, we recommend $\gamma \in [0.5, 2.0]$ as a reasonable range. Higher values provide more aggressive adaptation but risk over-coverage; lower values provide more conservative adaptation.

\subsection{Additional Robustness Checks}

\textbf{Calibration Split:} Results are stable across calibration fractions from 30\% to 70\%. The average improvement ranges from +9.0pp (at 70\% calibration) to +12.5pp (at 50\% calibration).

\textbf{Lookback Window:} Using 6-month or 24-month windows for the signal yields similar results (improvements of +11.2pp and +13.1pp respectively).

\textbf{Momentum Control:} Orthogonalizing the signal against momentum (to address concern that our proxy captures momentum rather than volatility) yields essentially identical results (+12.7pp improvement), suggesting momentum is not driving our findings.
