\documentclass[11pt,twocolumn]{article}

\usepackage[margin=1in]{geometry}
\usepackage{amsmath}
\usepackage{amssymb}
\usepackage{graphicx}
\usepackage{booktabs}
\usepackage{hyperref}
\usepackage{natbib}
\usepackage{times}

\title{\textbf{Not All Factors Crowd Equally:\\A Game-Theoretic Model of Alpha Decay}}

\author{Chorok Lee\\
\textit{KAIST}\\
\texttt{choroklee@kaist.ac.kr}}

\date{}

\begin{document}

\maketitle

\begin{abstract}
We present a game-theoretic model of factor alpha decay driven by strategy crowding.
Our model predicts that alpha decays hyperbolically as $\alpha(t) = K/(1 + \lambda t)$,
where $K$ is initial alpha capacity and $\lambda$ is the discovery rate.
Using Fama-French factor data from 1963-2024, we find that the model fits momentum factor
decay well ($R^2 = 0.65$, in-sample) and correctly predicts the direction of continued decay
out-of-sample (2016-2024). However, the model systematically over-predicts remaining alpha
(predicted: 0.30, actual: 0.15), suggesting crowding \textit{accelerated} beyond historical
rates after 2015---coinciding with the proliferation of factor ETFs and commission-free trading.
Crucially, the model fails for value (HML) and size (SMB) factors, which exhibit different
decay dynamics. We argue this reflects a fundamental distinction: \textit{mechanical} factors
(momentum) crowd quickly due to unambiguous signals, while \textit{judgment-based} factors
(value) crowd slowly due to definitional ambiguity. Our hyperbolic model outperforms linear
and exponential baselines for momentum ($R^2$: 0.65 vs. 0.51 vs. 0.61), validating the
game-theoretic foundation, while the model's failure on value factors is itself informative
about heterogeneous crowding dynamics.
\end{abstract}

\textbf{Keywords:} factor investing, alpha decay, crowding, game theory, momentum

%% =================================================================
\section{Introduction}
%% =================================================================

The momentum factor returned approximately 10\% annually in the 1990s. Today, that figure is closer to 2\%. What happened?

A growing body of evidence documents the decay of factor premia following academic publication \citep{mclean2016}. McLean and Pontiff found that approximately 50\% of anomaly alpha disappears post-publication, consistent with investors learning from research and arbitraging away returns. However, existing work largely documents decay descriptively rather than modeling it mechanistically.

We propose a game-theoretic model of alpha decay rooted in strategy crowding. The core insight is simple: when $N$ agents discover and trade the same profitable signal, they compete for a fixed ``alpha capacity'' $K$. In Nash equilibrium, each agent earns $\alpha_i = K/N$. As agents discover the signal over time following a Poisson process with rate $\lambda$, aggregate alpha decays as:
\begin{equation}
\alpha(t) = \frac{K}{1 + \lambda t}
\label{eq:decay}
\end{equation}

This hyperbolic decay model yields testable predictions. We fit it to momentum factor returns (1963-2024) and find $R^2 = 0.65$ with implied half-life of 5.5 years---consistent with the timing of momentum's publication \citep{jegadeesh1993} and subsequent institutional adoption.

More importantly, we test the model out-of-sample. Training on 1995-2015 and predicting 2016-2024, the model correctly predicts continued decay. However, it over-predicts remaining alpha (0.30 vs. 0.15 actual), suggesting crowding \textit{accelerated} post-2015. This prediction error is itself informative: it coincides with the democratization of factor investing through low-cost ETFs and commission-free trading.

A key finding emerges: \textbf{not all factors crowd equally}. The model fits momentum well but fails for value (HML) and size (SMB), which exhibit negative Sharpe ratios in recent periods. We attribute this to factor heterogeneity: momentum is a \textit{mechanical} signal (buy recent winners) that crowds quickly, while value requires \textit{judgment} (what is ``cheap''?) and crowds more diffusely.

%% =================================================================
\section{Related Work}
%% =================================================================

\textbf{Post-publication decay.} McLean and Pontiff \citep{mclean2016} documented that returns to 82 characteristics decline by approximately 50\% after publication, with stronger decay for strategies with lower arbitrage costs. Penasse \citep{penasse2017} formalized this with a model of investor learning. Our contribution is a game-theoretic micro-foundation that yields a specific functional form for decay.

\textbf{Market microstructure.} Kyle's \citep{kyle1985} seminal model shows that informed traders' profits depend on the number of informed agents. With $N$ informed traders, individual alpha scales as $1/N$. We extend this intuition to explain temporal decay as agents sequentially discover a signal.

\textbf{Factor crowding.} Recent work from practitioners documents rising factor crowding. Goldman Sachs reports hedge fund crowding at record highs, with 735 funds holding \$2.4T in equity positions. Chincarini et al. \citep{chincarini2024} find that crowding predicts future underperformance. Our model provides theoretical grounding for these observations.

\textbf{Strategy decay.} Capital Fund Management \citep{cfm2021} identifies three decay mechanisms: crowding, regime change, and data mining. We focus on crowding and show it produces hyperbolic (not exponential) decay under standard assumptions.

%% =================================================================
\section{Model}
%% =================================================================

\subsection{Setup}

Consider $N(t)$ agents who have discovered a profitable signal at time $t$. Each agent is small relative to the market (price-taker) but collectively they affect prices through market impact.

Let the signal predict excess return $r$ with edge $\alpha_0$ when undiscovered. Total alpha capacity is $K = \alpha_0/2$. Agents trade quantity $q_i$ and face price impact $\Delta P = \gamma \sum_i q_i$ (Kyle's lambda).

\subsection{Single-Period Equilibrium}

Each agent maximizes expected profit:
\begin{equation}
\max_{q_i} \mathbb{E}\left[q_i \cdot \left(r - \gamma \sum_j q_j\right)\right]
\end{equation}

In symmetric Nash equilibrium with $q_i = q^*$:
\begin{equation}
q^* = \frac{\alpha_0}{2\gamma N}
\end{equation}

Substituting back, equilibrium alpha per agent is:
\begin{equation}
\alpha_i = \frac{\alpha_0}{2N} = \frac{K}{N}
\end{equation}

\textbf{Result 1:} Alpha per agent decays as $1/N$---hyperbolic in the number of discoverers.

\subsection{Dynamic Model with Entry}

Agents discover the signal at rate $\lambda$ (Poisson process). Expected number at time $t$: $\mathbb{E}[N(t)] = \lambda t$. Substituting:
\begin{equation}
\alpha(t) = \frac{K}{1 + \lambda t}
\end{equation}

This is our hyperbolic decay model (Equation \ref{eq:decay}).

\subsection{Entry Equilibrium}

With entry cost $c$, agent $N+1$ enters if $K/(N+1) > c$. In equilibrium:
\begin{equation}
N^* = K/c, \quad \alpha_\infty = c
\end{equation}

\textbf{Result 2:} Long-run alpha equals entry cost (zero-profit condition). The factor doesn't die---it earns just enough to cover costs.

%% =================================================================
\section{Empirical Analysis}
%% =================================================================

\subsection{Data}

We use Fama-French factor returns from Kenneth French's data library (1963-2024, monthly). Factors include market (MKT-RF), size (SMB), value (HML), and momentum (Mom). We compute rolling 36-month Sharpe ratios as our alpha proxy.

\subsection{In-Sample Fit: Momentum}

Figure \ref{fig:decay} shows the rolling Sharpe ratio of momentum alongside our fitted model. We estimate $K = 1.66$ and $\lambda = 0.0145$ per month, yielding:
\begin{itemize}
    \item \textbf{In-sample $R^2$}: 0.65
    \item \textbf{Half-life}: $1/\lambda = 69$ months $\approx$ 5.7 years
\end{itemize}

The 5-7 year half-life is consistent with the timeline from Jegadeesh and Titman's 1993 publication to visible decay in the late 1990s.

\begin{figure}[h]
\centering
\includegraphics[width=0.95\columnwidth]{figures/fig3_model_fit.png}
\caption{Momentum factor: Rolling Sharpe ratio (blue) and fitted hyperbolic decay model (red). $R^2 = 0.65$.}
\label{fig:decay}
\end{figure}

\subsection{Out-of-Sample Prediction}

We train on 1995-2015 and predict 2016-2024 (Figure \ref{fig:oos}). Results:
\begin{itemize}
    \item \textbf{Direction correct}: Model predicts continued decay $\checkmark$
    \item \textbf{Magnitude}: Predicted Sharpe $\approx$ 0.30, Actual $\approx$ 0.15
    \item \textbf{RMSE}: 0.19
\end{itemize}

The systematic over-prediction is informative. A model trained on 1990-2015 captures that era's equilibrium decay rate. The over-prediction post-2015 suggests a structural break in crowding dynamics, coinciding with factor ETF proliferation and commission-free trading.

\begin{figure}[h]
\centering
\includegraphics[width=0.95\columnwidth]{figures/fig5_momentum_oos.png}
\caption{Out-of-sample prediction. Train: 1995-2015, Test: 2016-2024. Model predicts direction correctly but over-estimates remaining alpha.}
\label{fig:oos}
\end{figure}

\subsection{Baseline Comparison}

Does hyperbolic decay actually outperform naive alternatives? Table \ref{tab:baselines} compares our model against linear ($\alpha = a - bt$) and exponential ($\alpha = Ke^{-\lambda t}$) decay. For momentum, hyperbolic decay achieves $R^2 = 0.65$, outperforming linear (0.51) by 27\% and exponential (0.61) by 7\%. This validates the game-theoretic foundation: the $1/N$ structure from Nash equilibrium produces better fit than ad-hoc functional forms.

\begin{table}[h]
\centering
\caption{Model comparison (1995-2024). Bold = best fit. All models fail for HML/SMB ($R^2 < 0.2$).}
\label{tab:baselines}
\begin{tabular}{lcccccc}
\toprule
 & \multicolumn{2}{c}{Hyperbolic} & \multicolumn{2}{c}{Linear} & \multicolumn{2}{c}{Exponential} \\
Factor & $R^2$ & RMSE & $R^2$ & RMSE & $R^2$ & RMSE \\
\midrule
Mom & \textbf{0.65} & 0.26 & 0.51 & 0.31 & 0.61 & 0.28 \\
HML & 0.05 & 0.47 & \textbf{0.07} & 0.46 & 0.06 & 0.47 \\
SMB & 0.10 & 0.38 & \textbf{0.17} & 0.36 & 0.13 & 0.37 \\
\bottomrule
\end{tabular}
\end{table}

Notably, for HML and SMB, \textit{no} model fits well ($R^2 < 0.2$), with linear decay marginally best. This is not a failure of our approach---it reveals that these factors do not follow smooth decay dynamics, consistent with our mechanical vs.\ judgment taxonomy.

\subsection{Cross-Factor Comparison}

Figure \ref{fig:panel} visualizes model fit across factors. Momentum exhibits clear hyperbolic decay; value and size show erratic behavior inconsistent with any smooth decay model.

\begin{figure}[h]
\centering
\includegraphics[width=0.95\columnwidth]{figures/fig10_cross_factor_panel.png}
\caption{Cross-factor comparison: Hyperbolic (red) vs. linear (green) baseline. Momentum fits hyperbolic decay well; HML and SMB do not fit any smooth model.}
\label{fig:panel}
\end{figure}

\subsection{Why Do Factors Crowd Differently?}

We propose a taxonomy:
\begin{itemize}
    \item \textbf{Mechanical factors} (Momentum): Signal is unambiguous (``buy recent winners''). Easy to replicate $\rightarrow$ crowds quickly $\rightarrow$ fits decay model.
    \item \textbf{Judgment factors} (Value): Signal requires interpretation (``what is cheap?''). Multiple definitions $\rightarrow$ crowds diffusely $\rightarrow$ decay less predictable.
\end{itemize}

This explains why momentum fits our model while value does not. Value investing involves judgment calls about book value, intangibles, and accounting. Different investors implement ``value'' differently, leading to heterogeneous crowding.

%% =================================================================
\section{Discussion}
%% =================================================================

\textbf{Implications for practitioners.} Our model suggests that mechanical, well-published factors (momentum, low volatility) face the fastest decay and should be monitored for crowding. Judgment-based factors may offer more durable alpha but with less predictable dynamics.

\textbf{Crowding detection.} The prediction residual (actual minus predicted) may serve as a crowding acceleration detector. Large negative residuals indicate faster-than-expected decay, potentially signaling regime change in strategy adoption.

\textbf{Limitations.} Our model assumes homogeneous agents and continuous discovery. Reality involves heterogeneous capital, capacity constraints, and discrete events (ETF launches). Future work should incorporate these features.

\textbf{The prediction gap as signal.} The gap between predicted (0.30) and actual (0.15) momentum Sharpe is not model failure---it reveals that something changed. Factor investing became more accessible post-2015 through ETFs, robo-advisors, and commission-free trading. Our model captures equilibrium dynamics but not regime changes in discovery rate $\lambda$.

%% =================================================================
\section{Conclusion}
%% =================================================================

We presented a game-theoretic model of factor alpha decay that yields a testable hyperbolic decay functional form. The model fits momentum well ($R^2 = 0.65$) and predicts continued decay out-of-sample, though it over-estimates remaining alpha---revealing accelerated crowding post-2015. Crucially, the model fails for value and size factors, highlighting that not all factors crowd equally. We attribute this to a mechanical vs. judgment distinction in factor signals.

Our findings suggest that the death of alpha is not uniform: some factors crowd quickly and predictably, while others follow more complex dynamics. Crowding risk varies by factor type---mechanical factors crowd predictably, judgment factors do not.

\bibliographystyle{plainnat}
\begin{thebibliography}{10}

\bibitem[McLean and Pontiff(2016)]{mclean2016}
R.~D. McLean and J.~Pontiff.
\newblock Does academic research destroy stock return predictability?
\newblock \textit{Journal of Finance}, 71(1):5--32, 2016.

\bibitem[Jegadeesh and Titman(1993)]{jegadeesh1993}
N.~Jegadeesh and S.~Titman.
\newblock Returns to buying winners and selling losers: Implications for stock market efficiency.
\newblock \textit{Journal of Finance}, 48(1):65--91, 1993.

\bibitem[Kyle(1985)]{kyle1985}
A.~S. Kyle.
\newblock Continuous auctions and insider trading.
\newblock \textit{Econometrica}, 53(6):1315--1335, 1985.

\bibitem[Penasse(2017)]{penasse2017}
J.~Penasse.
\newblock Understanding alpha decay.
\newblock Working paper, University of Luxembourg, 2017.

\bibitem[CFM(2021)]{cfm2021}
Capital Fund Management.
\newblock Why and how systematic strategies decay.
\newblock Technical report, 2021.

\bibitem[Chincarini et al.(2024)]{chincarini2024}
L.~Chincarini, D.~Kim, and F.~Moneta.
\newblock Crowded spaces and anomalies.
\newblock Working paper, 2024.

\bibitem[Fama and French(1993)]{fama1993}
E.~F. Fama and K.~R. French.
\newblock Common risk factors in the returns on stocks and bonds.
\newblock \textit{Journal of Financial Economics}, 33(1):3--56, 1993.

\end{thebibliography}

\end{document}
